\subsection*{Materialization}


Materialization in a column-store system refers to the process of constructing tuples from the different attributes pulled from each of the columns. Determining at what time a system will materialize the tuples depends on the execution engine and the query optimizer.


Column-stores that store data on the disk in columns, but have a row-store execution engine, utilize early materialization. Because a row-store engine is used, tuples must be constructed before they are read into the engine. This implementation does not utilize the full benefits of column-store systems it cannot take advantage of vectorized processing as well as other column optimizations.


There are four advantages to late materialization\cite{colvsrow}:


\begin{enumerate}
	\item Some select and aggregate operators do not require tuple construction, so the materialization overhead can be completely avoided.
	\item Since data must be decompressed before tuples are reconstructed, late materialization allows for the operation on compressed data.
	\item Cache misses are reduced because irrelevant attributes are not being read into the CPU.
	\item Vectorized processing can be utilized.
\end{enumerate}


Late materialization is one of the main innovations of MonetDB. MonetDB takes an input of BATs and produces an output BAT. However, Monet DB requires full materialization of intermediate BAT results. After each operation on a column, Monet DB requires that identifies qualifying tuples\cite{monetcpu}. Vectorized processing is an improvement upon Monet DB provided by VectorWise that avoids the penalties incurred by full materialization\cite{vectorwise}. Vectorized processing will be discussed in the next section.


Another method that is more flexible, but also employs late materialization is demonstrated in Vertica and IBM BLU. These technologies utilize column-groups, which are columns that are multilple columns stored together on a page, like a mini table\cite{now}. Based on the access patterns of the data, these column groupings can be utilized.



