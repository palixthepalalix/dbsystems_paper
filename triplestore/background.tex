\subsection*{Background}


Graph databases are a collection of nodes and edges that are used to store data. A graph database does not require a key or identifier to link it to another record, it just uses edges to connect adjacent and related nodes. A relational database, in contrast, uses keys to relate values in one table to a value in another table, and finding the related values may require a scan of the entire table, whereas in a graph-based system, one need only to look at the nodes adjacent to the given node to find related nodes. Graph databases offer many optimizations for associative datasets.





One way in which graph data is represented is through RDF (Resource Description Framework). 


%3 phase structure
%loading
%iterative computing
%dumping
%heavily assosiated databases
%SPARQL and RDF
%property based